\begin{song}{Måsen}{riktigamasen}
\mel{När månen vandrar}
\begin{vers}
Det satt en mås på en klyvarbom \\
och tom i krävan var kräket. \\
Och tungan lådde vid skepparns gom,\\ 
där skutan låg uti blecket. \\
"Jag vill ha sill!" hördes måsen rope \\
och skepparn svarte: "Jag vill ha OP! \\
Om blott jag får, om blott jag får." \\
\end{vers}
 


  \begin{vers}
Nu lyfter måsen från klyvarbom, \\
och vinden spelar i tågen. \\
Och OP:n svalkat har skepparns gom, \\
jag önskar blott att jag såg'en. \\
Så nöjd och lycklig den arme saten, \\
han sätter storsegel den krabaten. \\
Till sjöss han far, och Halvan tar. \\
\end{vers}
 

  
  \begin{vers}
Den mås som satt på en klyvarbom,\\ 
den är nu död och begraven, \\
och skepparn som drack en flaska rom, \\
han har nu drunknat i haven. \\
Så kan det gå om man fått för mycket, \\
om man för brännvin har fattat tycke. \\
Vi som har kvar, vi resten tar. \\
\end{vers}
 


{\Large Dromedaren}\\{\tiny  Uppsala teknolog- och naturvetarkårs sångbok}
  \begin{vers}
Jag tänker sälja min dromedar.\\ 
Jag tänker flytta till norden. \\
Vem vill va’ bosatt uti ett land, \\
där man får ligga vid borden? \\
Nu konverterar jag här på snabben! \\
Jag vill ha akvavit till kebaben! \\
Var ingen mes, fyll upp din fez! \\
\end{vers}
 

  {\Large Månen}\\{\tiny  Uppsala teknolog- och naturvetarkårs sångbok}
  \begin{vers}
Nu månen vandrar sin tysta ban \\
och tittar in genom rutan. \\
Då tänker jag att på ljusan dag \\
då kan jag klara mig utan. \\
Då kan jag klara mig utan måne, \\
men utan renat och utan skåne, \\
det vete fan, det vete fan. \\
\end{vers}
 

\newpage
 {\Large Mesen}\\{\tiny  Tongångarnes bidrag till Riks-SMASK sångbok 1999}
  \begin{vers}
Det satt en mes i en klyvarmast, \\
där sågs han ragla och svaja. \\
För trots att frön var hans enda last \\
var han nu full som en kaja. \\
"Vad har du gjort!" hördes skepparn stöna \\
och mesen svarte: "Jag rökte fröna! \\
I egen holk, i egen holk." \\
\end{vers}
 

  {\Large Musen}\\{\tiny  Tongångarnes bidrag till Riks-SMASK sångbok 1999}
  \begin{vers}
Det satt en mus i en hushållsost \\
och åt och åt utan måtta \\
tills osten blev till en mushåls-ost \\
och han en klotformad råtta. \\
"Så bra", sa musen "att va en fettboll \\
nu kan jag rulla med hast åt rätt håll:\\ 
Ostindien, Ostindien."\\ 
\end{vers}
 

  {\Large Phaddergrupp 8}\\{\tiny  8:an såklart}
  \begin{vers}
Phaddergrupp 8:an är väldigt lat\\
Har inte bokat lokalen\\
Har inte fixat nån jävla mat\\
Vi byter ut den här raden!\\
Så kan det gå om man är en åttan\\
Då får man ta sig ett skott i råttan\\
Ett cykelställ, varenda kväll\\
\end{vers}
 

  {\Large Älgen}\\{\tiny  Tongångarnes bidrag till Riks-SMASK sångbok 1999}
  \begin{vers}
Det satt en älg i en klyvartopp, \\
förklädd i älgjaktens månad. \\
Han var befjädrad till horn och kropp \\
Ja, skepparn blev smått förvånad. \\
"Jag är en mås, goa skepparn", ljög den \\
förklädda älgen och sedan flög den. \\
Mjukt landa den, på skepparen. \\
\end{vers}
 

  {\Large JASen}\\{\tiny  Uppsala teknolog- och naturvetarkårs sångbok}
  \begin{vers}
Det flög en JAS över Västerbron \\
men styrsystemet var trasigt. \\
Piloten ut sköt sig med kanon \\
för planet svängde så knasigt. \\
"Jag vill ju uppåt, jag vill ju mer" \\
men planet svarte: "Jag vill ju ner" \\
"mot alla hjon på Västerbron." \\
\end{vers}
 
\newpage
  {\Large När nubben blänker}\\{\tiny  Uppsala teknolog- och naturvetarkårs sångbok}
  \begin{vers}
När nubben blänker i immigt glas \\
som hoppets strålande stjärna, \\
då är det avsett att det ska tas \\
förutan fruktan och gärna. \\
Så klang och klingom, så tar vi supen, \\
den läskar härligt den torra strupen. \\
Ja, skål gutår, ja, skål gutår! \\
\end{vers}
 

  {\Large Boten Anna}\\{\tiny  KTH:s Elektro}
  \begin{vers}
Min kompis Anna hon är en bot \\
Hon röjer upp i kanalen \\
Och varje gång jag hör hennes låt \\
Så får jag ont i analen\\ 
Jag är så trött på den jävla låten \\
Kan någon vänlig själ banna boten \\
Det vette fan, jag fick en ban! \\
\end{vers}


  {\Large Teknologen}\\{\tiny  KTH:s Elektro}
  \begin{vers}
Min kompis Bengt är en teknolog \\
och han dricker rätt mycke’. \\
För flickor är ingen lyckodrog \\
när man för punsch fattat tycke. \\
Han smakar gärna på någon flicka... s, \\
men han mår bäst när det börjar dricka... s. \\
Vem saknar kvinns, när punschen finns? \\
\end{vers}
 

  {\Large Räven}\\{\tiny  PQ}
  \begin{vers}
Det satt en räv i en vinterskog\\
Han var rätt sugen på rönnbär\\
Men rönnbärsbuskarna vintern tog\\
Och räven sade så här\\
Att trots snöns glitter, ja det sa räven \\
Är jag inte bitter, för de där bären\\
De är sura, de är sura\\
\end{vers}
 

  {\Large Mazur}
  \begin{vers}
Min kompis Mazur är rätt så stor\\
Och han är känd på Systemet\\
Han har körkort i mången färg\\
Och ett med lastbilsemblemet\\
Han kastar DP med bara händer\\
Han svinnar tre BIB:ar vin i sänder\\
Och han är stor, så jävla stor\\
\end{vers}
 
\newpage
  {\Large Grabben i graven bredvid}
  \begin{vers}
Det låg en grabb i en grav bredvid\\
Han hade gått på elektro\\
Han hade undrat utpå sitt liv\\
Hur ska man få nån respekt då\\
Han hade pillat med den där saken\\
Sen hade någon där dratt i spaken\\
Gnistorna slog, och han dog\\
\end{vers}
 

  {\Large Den vingklippta måsen}
  \begin{vers}
Det satt en mås på en klyvarbom \\
... \\
Jag vill ha OP! \\
\end{vers}
 

  {\Large Sara}\\{\tiny  Någon och John}
  \begin{vers}
Min kompis Sara är ingen brud\\
Han verkar alltid belåten\\
På festen kom som ett olycksbud\\
När bullarna låg på plåten\\
Han förvandla dom där på snabben\\
Till platta kakor det var ju tabben\\
Får du problem, de' Saras fel\\
\end{vers}

{\Large Dûscž$\hbar$kqĕ}\\{\tiny  ATM}
\begin{vers}
Min kompis Duszch är så jävla lost\\
går alltid vilse i Danmark\\
Han hade tur att det ej var frost\\
för han fick sova på bar mark\\
Så kan det gå om man är helt borta\\
för ens lokalsinne kom till korta\\
Vår lille man, fel väg tog han\\
\end{vers}

{\Large Hawaii matte}\\{\tiny  ATM,Dûscž$\hbar$kqĕ}
\begin{vers}

Chalmers är ganska lätt\\
man bara fuskar med matten\\
Wolfram Alpha varenda dag\\
så får vi sova på natten! \\
Nu ska du se hur vi deriverar\\
vi bara delar med ett $\mathrm{d}t$ här\\
Rigorösitet? Vad fan är deT! \\
\end{vers}

{\Large Farbror PQ}
\begin{vers}
Farbror PQ är speciell\\
tar alltid av sin sin tröja\\
Han kan bli rätt så sensuell\\
Han vill ju få alla nöjda\\
Ja vi går hem nu till farbror PQ\\
nu ska det supas, det fan det vet du\\
Fem komma två, det får väl gå
\end{vers}

\end{song}
